\documentclass[]{article}
\usepackage{lmodern}
\usepackage{amssymb,amsmath}
\usepackage{ifxetex,ifluatex}
\usepackage{fixltx2e} % provides \textsubscript
\ifnum 0\ifxetex 1\fi\ifluatex 1\fi=0 % if pdftex
  \usepackage[T1]{fontenc}
  \usepackage[utf8]{inputenc}
\else % if luatex or xelatex
  \ifxetex
    \usepackage{mathspec}
  \else
    \usepackage{fontspec}
  \fi
  \defaultfontfeatures{Ligatures=TeX,Scale=MatchLowercase}
\fi
% use upquote if available, for straight quotes in verbatim environments
\IfFileExists{upquote.sty}{\usepackage{upquote}}{}
% use microtype if available
\IfFileExists{microtype.sty}{%
\usepackage[]{microtype}
\UseMicrotypeSet[protrusion]{basicmath} % disable protrusion for tt fonts
}{}
\PassOptionsToPackage{hyphens}{url} % url is loaded by hyperref
\usepackage[unicode=true]{hyperref}
\hypersetup{
            pdftitle={Southeast Alaska Golden King Crab Harvest Strategy},
            pdfauthor={Andrew Olson and Katie Palof},
            pdfborder={0 0 0},
            breaklinks=true}
\urlstyle{same}  % don't use monospace font for urls
\usepackage[margin=1in]{geometry}
\usepackage{graphicx,grffile}
\makeatletter
\def\maxwidth{\ifdim\Gin@nat@width>\linewidth\linewidth\else\Gin@nat@width\fi}
\def\maxheight{\ifdim\Gin@nat@height>\textheight\textheight\else\Gin@nat@height\fi}
\makeatother
% Scale images if necessary, so that they will not overflow the page
% margins by default, and it is still possible to overwrite the defaults
% using explicit options in \includegraphics[width, height, ...]{}
\setkeys{Gin}{width=\maxwidth,height=\maxheight,keepaspectratio}
\IfFileExists{parskip.sty}{%
\usepackage{parskip}
}{% else
\setlength{\parindent}{0pt}
\setlength{\parskip}{6pt plus 2pt minus 1pt}
}
\setlength{\emergencystretch}{3em}  % prevent overfull lines
\providecommand{\tightlist}{%
  \setlength{\itemsep}{0pt}\setlength{\parskip}{0pt}}
\setcounter{secnumdepth}{0}
% Redefines (sub)paragraphs to behave more like sections
\ifx\paragraph\undefined\else
\let\oldparagraph\paragraph
\renewcommand{\paragraph}[1]{\oldparagraph{#1}\mbox{}}
\fi
\ifx\subparagraph\undefined\else
\let\oldsubparagraph\subparagraph
\renewcommand{\subparagraph}[1]{\oldsubparagraph{#1}\mbox{}}
\fi

% set default figure placement to htbp
\makeatletter
\def\fps@figure{htbp}
\makeatother

\usepackage{eso-pic,graphicx,transparent}
\usepackage{booktabs}
\usepackage{longtable}
\usepackage{array}
\usepackage{multirow}
\usepackage{wrapfig}
\usepackage{float}
\usepackage{colortbl}
\usepackage{pdflscape}
\usepackage{tabu}
\usepackage{threeparttable}
\usepackage{threeparttablex}
\usepackage[normalem]{ulem}
\usepackage{makecell}
\usepackage{xcolor}

\title{Southeast Alaska Golden King Crab Harvest Strategy}
\author{Andrew Olson and Katie Palof}
\date{07/21/2020}

\begin{document}
\maketitle

\AddToShipoutPictureFG{
  \AtPageCenter{% or \AtTextCenter
    \makebox[0pt]{\rotatebox[origin=c]{45}{%
      \scalebox{5}{\texttransparent{0.3}{DRAFT}}%
    }}
  }
}

\section*{Fishery Overview}\label{fishery-overview}
\addcontentsline{toc}{section}{Fishery Overview}

The Alaska Department of Fish and Game (Department) GKC fishery is a
data-limited fishery that is managed in Southeast using a 3-S management
system that is composed of a minimum legal size of 7.0 in/177.8 mm
carapace width (CW), male-only, and a season from February--November.
The majority of the harvest occurs in the commercial sector where the
fishery extends across 7 management areas (Northern, Icy Strait, North
Stephens Passage, East Central, Mid and Lower Chatham Strait, and
Southern). Annually the Department evaluates stock status and
establishes guideline harvest levels (GHLs) for each management area
using fishery dependent data (Olson, Siddon, and Eckert 2018, Stratman
et al. (2017)).

The GKC fishery rapidly developed after the collapse of the red and blue
king crab fisheries in the early 1980s with harvest subsequently harvest
peaking in the late 1980s and early 2010s while also experiencing a
period of collapse in the 1990s. Harvest has been steadily declining
since 2011 and many of the management areas are currently in a collapsed
state (Olson, Siddon, and Eckert 2018, Stratman et al. (2017), Stratman
(2020)).

\begin{table}[!h]

\caption{\label{tab:unnamed-chunk-1}Golden king crab guideline harvest ranges for Registration Area A [5 AAC 34.115]}
\centering
\resizebox{\linewidth}{!}{
\begin{tabular}[t]{l|l}
\hline
\rowcolor{gray}  \textcolor{white}{\textbf{Management.Area}} & \textcolor{white}{\textbf{Guideline.Harvest.Range..lbs.}}\\
\hline
\rowcolor{gray!6}  Northern & 0--175,000\\
\hline
Icy Strait & 0--75,000\\
\hline
\rowcolor{gray!6}  North Stephens Passage & 0--25,000\\
\hline
East Central & 0--300,000\\
\hline
\rowcolor{gray!6}  Mid-Chatham Strait & 0--150,000\\
\hline
Lower Chatham Strait & 0--50,000\\
\hline
\rowcolor{gray!6}  Southern & 0--25,000\\
\hline
\end{tabular}}
\end{table}

\subsection*{Biology}\label{biology}
\addcontentsline{toc}{subsection}{Biology}

Golden king crab are relatively long-lived slow growing species that
have an asychronous 20-month reproductive cycle, morphometric maturity
at approximately 8 years of age, lecithtrophic larvae that remain at
depth,and exhibit spatial variability in size at maturity across the
North Pacific and among the 7 management areas within Southeast Alaska
where size at matruity increases with increases in latitude. Certain
aspects of this species' life history are well documented whereas other
critical compenents such as, growth rates, age at maturity, longevity,
etc.) are unknown (Hebert et al. 2008, Christopher Long and Van Sant
(2016), Nizyaev (2005), Olson, Siddon, and Eckert (2018), Sloan (1985),
Somerton and Otto (1986)).

\subsection*{Proposed plan}\label{proposed-plan}
\addcontentsline{toc}{subsection}{Proposed plan}

Metrics to assess stock health and fishery performance. In-season
fishery perfomance data will be reviewed and analyzed bi-weekly to
inform harvest strategy decision rules in-season and determine GHLs
post-season.

\subsubsection*{Fishery Objective}\label{fishery-objective}
\addcontentsline{toc}{subsubsection}{Fishery Objective}

The primary objective of this harvest strategy is to rebuild the
Southeast Alaska GKC resource to sustainable levels while minimizing and
mitigating ecological risks from fishing related activites, maximizing
profitability for the commercial sector, monitoring social and economic
benefits of the fishery to the community, maintain various size and age
compositions of stocks in order to maintain longterm reproductive
viability, minimize handling and unnecessary mortality of non-legal GKC
and non-target species, and reducing dependency on annual recruitment.
These objectives remain consistent with Board of Fisheries' Policy on
King and Tanner Crab Resource Management (90-04-FB, March 23, 1990).

\subsection*{Performance Indicators}\label{performance-indicators}
\addcontentsline{toc}{subsection}{Performance Indicators}

The primary performance indicator used in this harvest strategy is
commercial catch rate defined as logbook catch per unit of effort (CPUE)
calculated from the CPUE at each logbook entry and averaged for the
entire season. Commercial catch rate is the number of of legal sized
male golden king crab per pot lift. Future iterations will incorporate
soak time component in order to standardize CPUE.

The secondary perfomance indicator used in this harvest strategy is
commercial catch rate obtained from fishticket where CPUE is calculated
from each harvest landing for the entire season divided by the
difference of the first and last catch date which is defined as active
fishing season. This secondary CPUE indicator is defined as pounds per
pot day. This is a similar metric as lbs/boat/day and provides
performance information that will be sued when GKC cannot be determined
if they were harvested as bycatch or were targeted during the concurrent
Tanner crab fishery.

Supplementary biological information by analyzing recruit class
distributions of GKC sampled during commercial landings. Size is defined
as the carapace length (CL) mm measurement and recruit class is used as
an indicator of shell age and is defined as recruit (new shell) and
post-recruit (old shell).

\subsection*{Reference Points}\label{reference-points}
\addcontentsline{toc}{subsection}{Reference Points}

The primary indicator Target Reference Point (RP\textsubscript{targ}) is
set at the average logbook CPUE for the years 2000--2017 as these years
capture when logbooks were required for the fishery in 2000 and
represents contrasting data (highs and lows) in fishery performance. The
Trigger Reference Point (RP\textsubscript{trig}) is set at 75\% of the
RP\textsubscript{targ} and the Limit Reference Point
(RP\textsubscript{lim}) is set at 50\% of the RP\textsubscript{targ}.

\subsection*{Monitoring Strategy}\label{monitoring-strategy}
\addcontentsline{toc}{subsection}{Monitoring Strategy}

\subsubsection*{Decision Rules}\label{decision-rules}
\addcontentsline{toc}{subsubsection}{Decision Rules}

The primary performance indicator (logbook CPUE rounded to the nearest
tenth) being the most readily available estimate of fishery performance
the folowing decision rules will be considered to guide in-season and
post-season management decions.

\begin{itemize}
\tightlist
\item
  \textbf{In-Season}

  \begin{itemize}
  \tightlist
  \item
    Fishery performance will be monitored bi-weekly and/or with a
    minimum of 500 pot lifts being required before taking management
    action whichever is the least restrictive under the following
    guidelines:

    \begin{itemize}
    \tightlist
    \item
      If logbook CPUE is \(\geq\) RP\textsubscript{targ} manage to GHL
      in-season;
    \item
      If logbook CPUE is \(\geq\) RP\textsubscript{trig} but \(<\)
      RP\textsubscript{targ} manage to GHL in-season and monitor
      closely;
    \item
      If logbook CPUE is \(\geq\) RP\textsubscript{lim} and \(<\)
      RP\textsubscript{trig} and less than the previous year CPUE close
      fishery early inseason, decrease GHL and personal use fishery bag
      and possession limits the following season;
    \item
      If logbook CPUE is \(<\) RP\textsubscript{lim} close fishery early
      inseason and subject to multiple year closure (1 year min.) for
      commercial and personal use fisheries the following season upon
      post-season review.
    \end{itemize}
  \end{itemize}
\end{itemize}

Due to the GKC and Tanner crab fishery occurring concurrently, permit
holders must indicate when they are targeting GKC, otherwise this can
bias CPUE low and trigger the aforementioned management actions.

\begin{itemize}
\tightlist
\item
  \textbf{Post-Season }

  \begin{itemize}
  \tightlist
  \item
    \textbf{Increase in the GHL}

    \begin{itemize}
    \tightlist
    \item
      If the most recent CPUE is \(>\) than the previous season and is
      \(>\) RP\textsubscript{targ} the GHL will increase up to a maximum
      of 20\% and not less than 10\% the following season;
    \item
      If the most recent CPUE is \(>\) than the previous season and
      \(\leq\) RP\textsubscript{targ} and \(>\) RP\textsubscript{trig}
      the GHL will remain the same or increase up to a maximum of 10\%
      the following season;
    \item
      If the most recent CPUE is \(>\) than the previous season and is
      \(\leq\) RP\textsubscript{trig} and \(>\) RP\textsubscript{limit}
      the GHL will will remain the same or increase up to a maximum of
      5\% the following season;
    \item
      New GHLs may not exceed respective managent area GHRs;
    \end{itemize}
  \item
    \textbf{Decrease in the GHL}

    \begin{itemize}
    \tightlist
    \item
      \textbf{If the fishery closed short of a GHL in-season due to poor
      fishery performance and/or the most recent CPUE is \(<\) than the
      previous season the GHL will be decreased based on the following
      conditions:}
    \item
      If CPUE is \(<\) the previous season and \(>\)
      RP\textsubscript{targ} the GHL will be reduced up to a maximum of
      20\% the following season;
    \item
      If CPUE is \(<\) than the previous season and is \(>\)
      RP\textsubscript{trig} and \(\leq\) RP\textsubscript{targ} the GHL
      will be reduced up to a maximum of 40\% the following season;
    \item
      If the fishery closed short in-season due to poor fishery
      performance and the most recent CPUE is \(<\) than the previous
      season and \(>\) RP\textsubscript{lim} than the previous season
      then the GHL decrease the following season will be within 20\% of
      the total harvest at the time of closure during the most recent
      season and not less than 5,000 lbs
    \end{itemize}
  \item
    \textbf{Closure and Re-opening}

    \begin{itemize}
    \tightlist
    \item
      The GHL will be equal to zero if logbook CPUE is \(<\) the
      RP\textsubscript{lim} warranting an area closure for a minimum of
      1 year and a maximum of 3 years.
    \item
      Upon re-opening an area after a closure the GHL will be equal to
      the harvest at the time of closure rounded to the nearest 1,000
      lbs or 5,000 lbs whichever is greatest
    \end{itemize}
  \item
    \textbf{No change in GHL}

    \begin{itemize}
    \tightlist
    \item
      None of the above conditions are met.
    \end{itemize}
  \end{itemize}
\item
  \textbf{Review of GHLs or Decision Rules}

  \begin{itemize}
  \tightlist
  \item
    If the the most recent CPUE is 40\% or more below the previous
    year's index, determine why decline occured and whether further
    management intervention is required to reduce the risk of localized
    depeletion; or
  \item
    If and when new information becomes available indicating that the
    assessment and GHL setting decision rules are not consistent with
    the Board's policy in managing the GKC resource sustainably, the
    decision rules must be reviewed and if apporpirate the reference
    points must be adjusted.
  \end{itemize}
\end{itemize}

\subsubsection*{Other Considerations for Management and Future
Recommendations}\label{other-considerations-for-management-and-future-recommendations}
\addcontentsline{toc}{subsubsection}{Other Considerations for Management
and Future Recommendations}

Logbook CPUE currently lacks a soak time data field and cannot be
standardized for comparsion across years. Soak time was introduced as a
reporting field in logbooks for the 2020 fishing season and will be used
in informing this harvest strategy in future iterations. In the interim
active fishing season may be looked at as a secondary performance
indicator to assess fishery performance in lieu of soak time and when
GKC harvest cannot be determined if the effort was targeted or bycatch
due to the concurrent GKC and Tanner crab fisheries.

This harvest strategy may be amended in future iterations as more
information and tools becomes avaible, drivers external to the
management of the fishery that increase the risk to crab stocks, or the
harvest strategy is not working effectively all of which should be
conducted collaboratively with industry. This harvest strategy provides
a first step in increasing transparency regarding management metrics
utilized in and post season and it is recommened that this harvest
strategy is further developed through the use of a management strategy
evaluation (MSE) to provided robustness and stability.

\subsection{References}\label{references}

\hypertarget{refs}{}
\hypertarget{ref-long2016}{}
Christopher Long, W, and Scott B Van Sant. 2016. ``Embryo Development in
Golden King Crab (Lithodes Aequispinus).'' \emph{Fishery Bulletin} 114
(1).

\hypertarget{ref-hebert2008}{}
Hebert, Kyle, W Davidson, Joe Stratman, Karla Bush, Gretchen Bishop,
Chris Siddon, Julie Bednarski, Adam Messmer, and Kellii Wood. 2008.
``2009 Report to the Alaska Board of Fisheries on Region 1 Shrimp, Crab,
and Scallop Fisheries.'' \emph{Alaska Department of Fish and Game,
Fishery Management Report}, no. 08-62. Anchorage.

\hypertarget{ref-nizyaev2005}{}
Nizyaev, SA. 2005. ``Biology of Golden King Crab (Lithodes Aequispinus
Benedict) Along the Islands of Kuril Ridge.'' \emph{Sakhalin Institute
of Fishery and Oceanography Publication, Yuzhno-Sakhalinsk (in
Russian)}.

\hypertarget{ref-olson2018spatial}{}
Olson, AP, CE Siddon, and GL Eckert. 2018. ``Spatial Variability in Size
at Maturity of Golden King Crab (Lithodes Aequispinus) and Implications
for Fisheries Management.'' \emph{Royal Society Open Science} 5 (3). The
Royal Society Publishing: 171802.

\hypertarget{ref-sloan1985}{}
Sloan, NA. 1985. ``Life History Characteristics of Fjord-Dwelling Golden
King Crabs Lithodes Aequispina.'' \emph{Marine Ecology Progress Series.
Oldendorf} 22 (3): 219--28.

\hypertarget{ref-somerton1986}{}
Somerton, David A, and RS Otto. 1986. ``Distribution and Reproductive
Biology of the Golden King Crab, Lithodes Aequispina, in the Eastern
Bering Sea.'' \emph{Fishery Bulletin} 84 (3). The Service: 571--84.

\hypertarget{ref-stratman2020}{}
Stratman, Joe. 2020. ``2019 Golden King Crab Stock Status and Management
Plan for the 2019/2020 Season.'' \emph{Alaska Department of Fish and
Game, Regional Information Report}, no. 1J20-11. Anchorage.

\hypertarget{ref-stratman2017}{}
Stratman, Joe, Tessa Bergmann, Kellii Wood, and Adam Messmer. 2017.
``Annual Management Report for the 2016/2017 Southeast Alaska/Yakutat
Golden King Crab Fisheries.'' \emph{Alaska Department of Fish and Game,
Fishery Management Report}, no. 17-57. Anchorage.

\section{Area Reports}\label{area-reports}

\subsection{Northern}\label{northern}

\subsubsection{Season Overview}\label{season-overview}

The Northern management area was closed for the 2020 season.

\begin{figure}[H]
\includegraphics[width=500px]{../figures/2020/Northern_harvest} \caption{Commercial fishery harvest from the Northern management area. Dots represents the GHL in a given year (2001–Present)}\label{fig:unnamed-chunk-2}
\end{figure}

\subsubsection{Reference Points}\label{reference-points-1}

The primary indicator Target Reference Point (RP\textsubscript{targ}) is
set at the average logbook CPUE (2.7) for the years 2000--2017 as these
years capture when logbooks were required for the fishery in 2000 and
represents contrasting data (highs and lows) in fishery performance. The
Trigger Regerence Point (RP\textsubscript{trig}) is set at 75\% of the
RP\textsubscript{targ} (2.0) and the Limit Reference Point
(RP\textsubscript{lim}) is set at 50\% of the RP\textsubscript{targ}
(1.3).

\begin{table}[!h]

\caption{\label{tab:unnamed-chunk-3}Golden King Crab logbook catch per unit effort (CPUE) reference points}
\centering
\begin{tabular}[t]{l|l|l}
\hline
\textcolor{black}{\textbf{Indicators}} & \textcolor{black}{\textbf{Reference.Point}} & \textcolor{black}{\textbf{Description}}\\
\hline
\rowcolor{gray!6}  \rowcolor{olive}  \textcolor{white}{\textbf{Target Reference Point}} & \textcolor{white}{\textbf{2.7 crab/pot}} & \textcolor{white}{\textbf{Average Commercial Logbook CPUE from 2000-2017}}\\
\hline
\rowcolor{orange}  \textcolor{white}{\textbf{Trigger Reference Point}} & \textcolor{white}{\textbf{2.0 crab/pot}} & \textcolor{white}{\textbf{75\% of the Target Reference Point}}\\
\hline
\rowcolor{gray!6}  \rowcolor{red}  \textcolor{white}{\textbf{Limit Reference Point}} & \textcolor{white}{\textbf{1.3 crab/pot}} & \textcolor{white}{\textbf{50\% of the Target Reference Point}}\\
\hline
\end{tabular}
\end{table}

\begin{figure}[H]
\includegraphics[width=500px]{../figures/2020/Northern logbook_cpue} \caption{Northern golden king crab reference points (Target, Trigger, and Limit) utilizing logbook CPUE. Dotted line represents a 3-year rolling average}\label{fig:unnamed-chunk-4}
\end{figure}

\subsection{Icy Strait}\label{icy-strait}

\subsubsection{Season Overview}\label{season-overview-1}

\begin{figure}[H]
\includegraphics[width=500px]{../figures/2020/Icy Strait_harvest} \caption{Commercial fishery harvest from the Icy Strait management area. Dots represents the GHL in a given year (2001–Present)}\label{fig:unnamed-chunk-5}
\end{figure}

\subsubsection{Reference Points}\label{reference-points-2}

\begin{table}[!h]

\caption{\label{tab:unnamed-chunk-6}Golden King Crab logbook catch per unit effort (CPUE) reference points}
\centering
\begin{tabular}[t]{l|l|l}
\hline
\textcolor{black}{\textbf{Indicators}} & \textcolor{black}{\textbf{Reference.Point}} & \textcolor{black}{\textbf{Description}}\\
\hline
\rowcolor{gray!6}  \rowcolor{olive}  \textcolor{white}{\textbf{Target Reference Point}} & \textcolor{white}{\textbf{2.2 crab/pot}} & \textcolor{white}{\textbf{Average Commercial Logbook CPUE from 2000-2017}}\\
\hline
\rowcolor{orange}  \textcolor{white}{\textbf{Trigger Reference Point}} & \textcolor{white}{\textbf{1.6 crab/pot}} & \textcolor{white}{\textbf{75\% of the Target Reference Point}}\\
\hline
\rowcolor{gray!6}  \rowcolor{red}  \textcolor{white}{\textbf{Limit Reference Point}} & \textcolor{white}{\textbf{1.1 crab/pot}} & \textcolor{white}{\textbf{50\% of the Target Reference Point}}\\
\hline
\end{tabular}
\end{table}

\begin{figure}[H]
\includegraphics[width=500px]{../figures/2020/Icy Strait logbook_cpue} \caption{Icy Strait golden king crab reference points (Target, Trigger, and Limit) utilizing logbook CPUE. Dotted line represents a 3-year rolling average}\label{fig:unnamed-chunk-7}
\end{figure}

\subsection{North Stephens Passage}\label{north-stephens-passage}

\subsubsection{Season Overview}\label{season-overview-2}

\begin{figure}[H]
\includegraphics[width=500px]{../figures/2020/North Stephens Passage_harvest} \caption{Commercial fishery harvest from the North Stephens Passage management area. Dots represents the GHL in a given year (2001–Present)}\label{fig:unnamed-chunk-8}
\end{figure}

\subsubsection{Reference Points}\label{reference-points-3}

\begin{table}[!h]

\caption{\label{tab:unnamed-chunk-9}Golden King Crab logbook catch per unit effort (CPUE) reference points}
\centering
\begin{tabular}[t]{l|l|l}
\hline
\textcolor{black}{\textbf{Indicators}} & \textcolor{black}{\textbf{Reference.Point}} & \textcolor{black}{\textbf{Description}}\\
\hline
\rowcolor{gray!6}  \rowcolor{olive}  \textcolor{white}{\textbf{Target Reference Point}} & \textcolor{white}{\textbf{1.6 crab/pot}} & \textcolor{white}{\textbf{Average Commercial Logbook CPUE from 2001-2017}}\\
\hline
\rowcolor{orange}  \textcolor{white}{\textbf{Trigger Reference Point}} & \textcolor{white}{\textbf{1.2 crab/pot}} & \textcolor{white}{\textbf{75\% of the Target Reference Point}}\\
\hline
\rowcolor{gray!6}  \rowcolor{red}  \textcolor{white}{\textbf{Limit Reference Point}} & \textcolor{white}{\textbf{0.8 crab/pot}} & \textcolor{white}{\textbf{50\% of the Target Reference Point}}\\
\hline
\end{tabular}
\end{table}

\begin{figure}[H]
\includegraphics[width=500px]{../figures/2020/North Stephens Passage logbook_cpue} \caption{North Stephens Passage golden king crab reference points (Target, Trigger, and Limit) utilizing logbook CPUE. Dotted line represents a 3-year rolling average}\label{fig:unnamed-chunk-10}
\end{figure}

\subsection{East Central}\label{east-central}

\subsubsection{Season Overview}\label{season-overview-3}

The East Central management area was closed for the 2020 season.

\begin{figure}[H]
\includegraphics[width=500px]{../figures/2020/East Central_harvest} \caption{Commercial fishery harvest from the East Central management area. Dots represents the GHL in a given year (2001–Present)}\label{fig:unnamed-chunk-11}
\end{figure}

\subsubsection{Reference Points}\label{reference-points-4}

\begin{table}[!h]

\caption{\label{tab:unnamed-chunk-12}Golden King Crab logbook catch per unit effort (CPUE) reference points}
\centering
\begin{tabular}[t]{l|l|l}
\hline
\textcolor{black}{\textbf{Indicators}} & \textcolor{black}{\textbf{Reference.Point}} & \textcolor{black}{\textbf{Description}}\\
\hline
\rowcolor{gray!6}  \rowcolor{olive}  \textcolor{white}{\textbf{Target Reference Point}} & \textcolor{white}{\textbf{3.4 crab/pot}} & \textcolor{white}{\textbf{Average Commercial Logbook CPUE from 2000-2017}}\\
\hline
\rowcolor{orange}  \textcolor{white}{\textbf{Trigger Reference Point}} & \textcolor{white}{\textbf{2.5 crab/pot}} & \textcolor{white}{\textbf{75\% of the Target Reference Point}}\\
\hline
\rowcolor{gray!6}  \rowcolor{red}  \textcolor{white}{\textbf{Limit Reference Point}} & \textcolor{white}{\textbf{1.7 crab/pot}} & \textcolor{white}{\textbf{50\% of the Target Reference Point}}\\
\hline
\end{tabular}
\end{table}

\begin{figure}[H]
\includegraphics[width=500px]{../figures/2020/East Central logbook_cpue} \caption{East Central golden king crab reference points (Target, Trigger, and Limit) utilizing logbook CPUE. Dotted line represents a 3-year rolling average}\label{fig:unnamed-chunk-13}
\end{figure}

\subsection{Mid-Chatham Strait}\label{mid-chatham-strait}

\subsubsection{Season Overview}\label{season-overview-4}

The Mid-Chatham Strait management area was closed for the 2020 season.

\begin{figure}[H]
\includegraphics[width=500px]{../figures/2020/Mid-Chatham_harvest} \caption{Commercial fishery harvest from the Mid-Chatham Strait management area. Dots represents the GHL in a given year (2001–Present)}\label{fig:unnamed-chunk-14}
\end{figure}

\subsubsection{Reference Points}\label{reference-points-5}

\begin{table}[!h]

\caption{\label{tab:unnamed-chunk-15}Golden King Crab logbook catch per unit effort (CPUE) reference points}
\centering
\begin{tabular}[t]{l|l|l}
\hline
\textcolor{black}{\textbf{Indicators}} & \textcolor{black}{\textbf{Reference.Point}} & \textcolor{black}{\textbf{Description}}\\
\hline
\rowcolor{gray!6}  \rowcolor{olive}  \textcolor{white}{\textbf{Target Reference Point}} & \textcolor{white}{\textbf{3.4 crab/pot}} & \textcolor{white}{\textbf{Average Commercial Logbook CPUE from 2000-2017}}\\
\hline
\rowcolor{orange}  \textcolor{white}{\textbf{Trigger Reference Point}} & \textcolor{white}{\textbf{2.5 crab/pot}} & \textcolor{white}{\textbf{75\% of the Target Reference Point}}\\
\hline
\rowcolor{gray!6}  \rowcolor{red}  \textcolor{white}{\textbf{Limit Reference Point}} & \textcolor{white}{\textbf{1.7 crab/pot}} & \textcolor{white}{\textbf{50\% of the Target Reference Point}}\\
\hline
\end{tabular}
\end{table}

\begin{figure}[H]
\includegraphics[width=500px]{../figures/2020/Mid-Chatham logbook_cpue} \caption{Mid-Chatham Strait golden king crab reference points (Target, Trigger, and Limit) utilizing logbook CPUE. Dotted line represents a 3-year rolling average}\label{fig:unnamed-chunk-16}
\end{figure}

\subsection{Lower Chatham Strait}\label{lower-chatham-strait}

\subsubsection{Season Overview}\label{season-overview-5}

The Lower Chatham Strait management area was closed for the 2020 season.

\begin{figure}[H]
\includegraphics[width=500px]{../figures/2020/Lower Chatham_harvest} \caption{Commercial fishery harvest from the Lower Chatham Strait management area. Dots represents the GHL in a given year (2001–Present)}\label{fig:unnamed-chunk-17}
\end{figure}

\subsubsection{Reference Points}\label{reference-points-6}

\begin{table}[!h]

\caption{\label{tab:unnamed-chunk-18}Golden King Crab logbook catch per unit effort (CPUE) reference points}
\centering
\begin{tabular}[t]{l|l|l}
\hline
\textcolor{black}{\textbf{Indicators}} & \textcolor{black}{\textbf{Reference.Point}} & \textcolor{black}{\textbf{Description}}\\
\hline
\rowcolor{gray!6}  \rowcolor{olive}  \textcolor{white}{\textbf{Target Reference Point}} & \textcolor{white}{\textbf{3.2 crab/pot}} & \textcolor{white}{\textbf{Average Commercial Logbook CPUE from 2000-2012}}\\
\hline
\rowcolor{orange}  \textcolor{white}{\textbf{Trigger Reference Point}} & \textcolor{white}{\textbf{2.4 crab/pot}} & \textcolor{white}{\textbf{75\% of the Target Reference Point}}\\
\hline
\rowcolor{gray!6}  \rowcolor{red}  \textcolor{white}{\textbf{Limit Reference Point}} & \textcolor{white}{\textbf{1.6 crab/pot}} & \textcolor{white}{\textbf{50\% of the Target Reference Point}}\\
\hline
\end{tabular}
\end{table}

\begin{figure}[H]
\includegraphics[width=500px]{../figures/2020/Lower Chatham logbook_cpue} \caption{Lower Chatham Strait golden king crab reference points (Target, Trigger, and Limit) utilizing logbook CPUE. Dotted line represents a 3-year rolling average}\label{fig:unnamed-chunk-19}
\end{figure}

\subsection{Southern}\label{southern}

\subsubsection{Season Overview}\label{season-overview-6}

\begin{figure}[H]
\includegraphics[width=500px]{../figures/2020/Southern_harvest} \caption{Commercial fishery harvest from the Southern management area. Dots represents the GHL in a given year (2001–Present)}\label{fig:unnamed-chunk-20}
\end{figure}

\subsubsection{Reference Points}\label{reference-points-7}

\begin{table}[!h]

\caption{\label{tab:unnamed-chunk-21}Golden King Crab logbook catch per unit effort (CPUE) reference points}
\centering
\begin{tabular}[t]{l|l|l}
\hline
\textcolor{black}{\textbf{Indicators}} & \textcolor{black}{\textbf{Reference.Point}} & \textcolor{black}{\textbf{Description}}\\
\hline
\rowcolor{gray!6}  \rowcolor{olive}  \textcolor{white}{\textbf{Target Reference Point}} & \textcolor{white}{\textbf{4.1 crab/pot}} & \textcolor{white}{\textbf{Average Commercial Logbook CPUE from 2000-2017}}\\
\hline
\rowcolor{orange}  \textcolor{white}{\textbf{Trigger Reference Point}} & \textcolor{white}{\textbf{3.1 crab/pot}} & \textcolor{white}{\textbf{75\% of the Target Reference Point}}\\
\hline
\rowcolor{gray!6}  \rowcolor{red}  \textcolor{white}{\textbf{Limit Reference Point}} & \textcolor{white}{\textbf{2.0 crab/pot}} & \textcolor{white}{\textbf{50\% of the Target Reference Point}}\\
\hline
\end{tabular}
\end{table}

\begin{figure}[H]
\includegraphics[width=500px]{../figures/2020/Southern logbook_cpue} \caption{Southern golden king crab reference points (Target, Trigger, and Limit) utilizing logbook CPUE. Dotted line represents a 3-year rolling average}\label{fig:unnamed-chunk-22}
\end{figure}

\end{document}
